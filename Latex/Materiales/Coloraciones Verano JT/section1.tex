\section{El tablero de Ajedrez}

En 1961, un F\'isico Te\'orico Brit\'anico llamado M. Fischer resolvi\'o un problema de tableros famoso de la \'epoca: ¿De cu\'antas formas es posible colocar $32$ domin\'os de $2\times 1$ en un tablero de ajedrez de $8\times 8$?. La respuesta que obtuvo es $12,988,816$ formas. 
Ahora, ¿Que sucede que cortamos dos casillas de esquinas opuestas?

\begin{ejem}
	 ¿De cu\'antas formas podemos poner $31$ domin\'os en el nuevo tablero?
\end{ejem}
\begin{proof}
	Este problema parece considerablemente m\'as dif\'icil, a simple vista. Sin embargo, ese no es el caso. Consideremos el tablero de Ajedrez, pintado en blanco y negro de manera alternada. Las dos casillas recortadas son del mismo color (pueden ser negras o blancas, pero siempre del mismo color). Por lo tanto de las $64$ casillas, hay $32$ de un color y $30$ de otro color. 
	
	Primero observemos que cada domin\'o cubre exactamente una casilla negra y una blanca. Suponiendo que el tablero pueda cubrirse con esas $31$ piezas, entonces deber\'ia haber $31$ casillas de cada color. Pero eso no ocurre. Concluimos entonces que es imposible cubrir el nuevo tablero con $31$ piezas de domin\'o. 
\end{proof}
 

Existen dos observaciones importantes que se pueden rescatar de \'esta prueba. 
\begin{enumerate}
	\item Podemos encontrarnos en dos situaciones diferentes en problemas de colocar piezas en tableros: O es posible cubrir el tablero o no. En el primer caso, es suficiente con encontrar una configuraci\'on para concluir que es posible, pero notemos que para concluir que es \textbf{imposible} cubrir el tablero, deber\'iamos mostrar que \textbf{ninguna} configuraci\'on funciona. Esto \'ultimo generalmente no es el camino correcto. Se necesita de una idea inteligente para atacar el problema. Una idea posible es la coloraci\'on.
	\item Existe una receta que ayuda\footnote{¡Ninguna t\'ecnica solucionar\'a todos los problemas!} a resolver muchos problemas de \'este tipo:
	\begin{enumerate}[a.]
		\item Colore\'a el tablero de una forma que te parezca \'util (Ajedrez). %Porque es util?
		\item Encontr\'a una propiedad que te d\'e la coloraci\'on (Cada domin\'o cubre exactamente una casilla de cada color). 
		\item Supon\'e que la hip\'otesis es falsa, y lleg\'a a una contradicci\'on (Deber\'ia haber $31$ casillas de cada color, pero eso no pasa).
	\end{enumerate}

	
	
\end{enumerate}

Veamos esta receta en acci\'on

\begin{ejem} \label{ejemplotetramino}
	Decidir si es posible cubrir completamente\footnote{A menos  que se especifique lo contrario, no se est\'a permitido de superponer las piezas ni salirse del tablero, pero podemos rotar y girar las piezas.} un tablero de $10 \times 10 $ con piezas iguales al de la figura.
\end{ejem}
\begin{proof}
	Lo primero es lo primero: intentar cubrir el tablero con las piezas indicadas. Luego de unos intentos fallidos, nuestra intuici\'on probablemente nos diga que no es posible cumplir el cometido. Es ah\'i cuando empezamos a poner las manos en la masa:
	
	\begin{enumerate}[a.]
		
		\item \textit{Colore\'a el tablero}
		
		El segundo ejemplo no puede ser diferente al primero, necesitamos empaparnos de esta nueva t\'ecnica: Coloreamos el tablero como un tablero de ajedrez. Esta parte ya la conocemos.
		
		\item \textit{Encontrar una propiedad en la coloraci\'on.}
		
		Podemos observar que, no importa donde se coloque la pieza, siempre estar\'a sobre tres casillas del mismo color y una casilla del color diferente. Es decir, cualquier pieza que se coloque sobre el tablero ocupar\'a una cantidad impar de casillas blancas. Adem\'as, $50$ de las $100$ casillas del tablero son blancas.
		
		\item \textit{Supongamos que la hip\'otesis es falsa}
		
		Es decir, es posible cubrir el tablero completamente, para lo cual necesitaremos $\frac{100}{4} = 25$ piezas. La observaci\'on clave es combinar los dos \'ultimos datos. Estas $25$ piezas cubrir\'an indefectiblemente una cantidad impar (suma de una cantidad impar de n\'umeros impares) de casillas blancas, pero hay $50$ casillas blancas, el cual es un n\'umero par. Luego, es imposible cubrir el tablero.
	\end{enumerate}
\end{proof}

\subsection{Problemas}

\begin{enumerate}
	\item En un sal\'on de clase est\'an sentados los alumnos formando un arreglo rectangular de 5  7. La maestra
	que quiere hacer una din\'amica, les pide a todos los alumnos que intercambien de lugar con un compañero
	vecino, movi\'endose un lugar ya sea a la izquierda, a la derecha, adelante o atr\'as de su lugar. Pepito, que
	sabe de matem\'aticas, le dice a la maestra que esto es imposible. ¿Por qu\'e tiene raz\'on Pepito?
	\item Un rat\'on se quiere comer una pila de cubitos de queso de $5\times5\times5$ y quiere ir comiendo cada cubito iniciando por un cubito de la orilla y
	terminando en el cubito central. Adem\'as, cada que come un cubito, el siguiente cubito que se come es uno
	de los adyacentes (no en diagonal). ¿Podr\'a el rat\'on comerse el queso de esta manera?
	
	\item El ejemplo \ref{ejemplotetramino} muestra el problema de cubrir un tablero de $10\times 10$ con un tetramin\'o (una pieza que se compone de cuatro casillas). Existen otros cuatro tipos de tetramin\'os (Si, esos que aparecen en el tetris). Analice cada caso: ¿Es posible cubrir el tablero $10\times 10$
	
	\item Una tripleta ordenada de n\'umeros enteros positivos $(a, b, n)$ es dicha \textit{veraniega} si es posible cubrir un cuadrado de $n \times n$ con piezas de tamaño $a\times b$. Halle la cantidad de tripletas \textit{veraniegas} de la forma $(a, b, n)$, donde $n$ equivale al año en el cual nos encontramos.
\end{enumerate}
