\section{Funcion Phi de Euler}

\begin{sol}
	\begin{enumerate}
		\item $\varphi (2^k) = 2^k-2^{k-1} = 2^{k-1}$
		\item Vea el siguiente item
		\item $\varphi(p^{k}) = p^{k}-p^{k-1}$
		\end{enumerate}
\end{sol}

\begin{sol}
	$\varphi(pq) = (p-1)(q-1)$
\end{sol}

\begin{sol}
	Ejercicio para el lector
\end{sol}

\begin{sol}
	Vea que $\varphi(n) = \prod_{i=1}^{r} p_{i}^{e_{i}-1}(p_{i}-1)$. Si hay alg\'un $p_{i}$ que es impar, $p_{i}-1$ es par, y caso contrario si $p_{1}=2, r = 1$ entonces $\varphi(n) = 2^{e_{1}-1}$. Como $n\geq 3 \implies e_{i}\geq 2 \implies 2|\varphi(n)$
\end{sol}

\begin{sol}
	Note que $(k, n) = n-k, n)$, debido al algoritmo de Euler. Sea $S_{n}$ la suma que queremos encontrar. Tenemos por lo tanto que 
	\begin{equation}
	2S_{n} = 2 \sum\limits_{\substack{(n, k) = 1 \\ 1\leq k \leq n}} k = \sum\limits_{\substack{(n, k) = 1 \\ 1\leq k \leq n}} k + \sum\limits_{\substack{(n, k) = 1 \\ 1\leq k \leq n}} n-k = \sum\limits_{\substack{(n, k) = 1 \\ 1\leq k \leq n}} n = n \cdot \varphi(n)
	\end{equation}
	Concluyendo que $S_{n} = \frac{n \cdot \varphi(n)}{2}$
\end{sol}



\begin{sol}
	Sea $\varphi(a) = \prod_{i=1}^{r} p_{i}^{e_{i}-1}(p_{i}-1), \varphi(b) = \prod_{j=1}^{k} p_{j}^{v_{j}-1}(p_{j}-1)$. Como $a|b\implies (r\leq k) \land ( e_{i} \leq v_{i}, \forall i\leq r)$. Por lo tanto, $\varphi(a)|\varphi(b)$.
\end{sol}

\begin{sol}
	Veamos que si $\varphi(n) = 4$, entonces ning\'un primo mayor a $5$ puede dividir a $n$. Si $5|n$, entonces $n=5$. Luego, solo $2$ y $3$ dividen a $n= 2^{a} 3^{b}$. Luego $\varphi(n) = 2^{a}\cdot 3^{b-1}$. Vemos que $b = 1$, luego $n = 4\cdot 3 = 12$. Si $b = 0$, entonces $a=3 \implies n = 8$. Las soluciones son entonces $n = 8, 12$. (Falta revisar)
\end{sol}

\begin{sol}
	Aplicaremos Inducci\'on sobre n. Se pueden checar valores chicos, para corroborar el caso base. Por ejemplo, $n = 1, 2, 3, 4, 5$. Supongamos que la f\'ormula se cumple para $n$, y sea $q$ un primo cualquiera. Queremos probar que la f\'ormula es satisfecha para $nq$.

	Sea $v$ la m\'axima potencia de $q$ que divide a $n$. $\sum_{d|nq} \varphi(d) = \sum_{d|n, d\not{|} q^{v}}\varphi(n) +\sum_{q^{v} |i} \varphi(q\cdot i) = n + \sum_{i|\frac{n}{q^{v+1}}} \varphi(q^{v})\varphi(i) = n + \varphi(q^v+1) (\sum_{i|\frac{n}{q^{v}}}) \varphi(i) = n + q^{v}(q-1)\frac{n}{q^{v}}= n + (q-1)n = nq. \square$
\end{sol}

