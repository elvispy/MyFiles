\section{Aritm\'etica Modular}

\subsection{Congruencias}

\begin{sol}
	Escoja $3$ y $-21$, por ejemplo.
\end{sol}

\begin{sol}
	\begin{enumerate}
		\item $(1\equiv 4 \mod 3) \land (4\equiv 2 \mod 2) $ no implica $1\equiv 2 \mod 2$.
		\item $2\cdot 12 \equiv 2\cdot 3 \mod 6$ no implica $ 12\equiv 3 \mod 6$.
		\item $1^2 \equiv 2^2 \mod 3$.
	\end{enumerate}
\end{sol}

\begin{sol}
	
	\begin{enumerate}
		
		\item $2^7 \equiv 5 (\mod 41) \implies (2^7)^3 \equiv 5^3 \equiv 125 \equiv 2 \implies 2^{21}\cdot  21 \equiv 2\cdot21 \equiv 1 \implies 2^{20} \cdot2\cdot21 \equiv 2^{20} \equiv 1 (\mod 41) \square$
		\item $20^3 \equiv 8000 \equiv 9 (\mod 61) \implies  20^6 \equiv 9^2 \equiv 20 \implies 20^{15} \equiv (20^6)^2  \cdot 20^3  \equiv 20^2 \cdot 9 \equiv 34\cdot 9 \equiv 306 \equiv 1 (\mod 61) \square$
		\item $2^6 \equiv -1 (\mod 13) \implies 2^{12} \equiv 1 \implies 2^{60} \equiv 1 \implies 2^{70} \equiv 2^{10} \equiv -2^{4} \equiv -16 \equiv -3$. Del mismo modo, tenemos que $3^3 \equiv 1 (\mod 13) \implies 3^{69} \equiv 1 \implies 3^{70} \equiv 3 \implies 2^{70} + 3^{70} \equiv 0 \square$
	
	\end{enumerate}
\end{sol}

\begin{sol} %Ejercicio 1.4
	Note que $12|k!$, $\forall k\geq 4$. Luego $\sum_{i=1}^{99}i! \equiv 1!+2!+3! \mod 12 \equiv 1+2+6 \equiv 9$.
\end{sol}

\begin{sol}
	Sea $k = \overline{a_{1}a_{2}\dots a_{n}}$ un numero cualquiera. Entonces $k = \sum_{i = 1}^{n} 10^{i-1} \cdot a_{i}$. Note que $10^{2j} \equiv 1 \mod 11 $ y $10^{2j+1} \equiv -1 \mod 11$, luego $k \equiv \sum_{i=1}^{n}(-1)^{i-1} \cdot a_{i}$. Lo cual representa la suma de las cifras en posiciones pares menos la suma de los numeros en posiciones impares.
\end{sol}


\begin{sol}
	$a^2 \equiv 1 \mod 24 \iff (a-1)(a+1) \equiv 0 \mod 24$. Pero como $a$ es impar, tanto $a+1$ como $a-1$ son pares y uno de ellos es multiplo de 4. Finalmente, como $a$ no es multiplo de 3, uno de los numeros $(a+1) o (a-1)$ es multiplo de 3. Eso implica que $(a+1)(a-1) \equiv 0 \square$.
\end{sol}

\begin{sol}
	Simplemente note que $a-b |a^{k}-b^{k}$ (Caso conocido de factorizaci\'on)
\end{sol}

\begin{sol}
	An\'alogo al ejercicio anterior.
\end{sol}

\begin{sol}
	Haciendo cuentas se puede llegar a que $2^{24} \equiv -1 \mod 97$. Lo cual implica que $2^{48} \equiv 1$. Vea que $2^4 \equiv 16 \mod 48$, $2^5 \equiv 32$, $2^6 \equiv 16$, y el ciclo se repite. Luego, como $2011$ es impar, $2^{2011} \equiv 32 \mod 48$. Eso significa que $2^{2011} = 48k +32 \implies 2^{2^{2011}} \equiv 2^{48k+32} \mod 97 \equiv 2^{32} \equiv 2^{24} \cdot 2^8 \equiv -256 \equiv 35 \square$
\end{sol}

\begin{sol}
	Note que si $k = 3q$, la cifra de las decenas de $k$ solo se ver\'a afectada por la cifra de las decentas de $q$ y las ciras de las unidades de $q$. Procedemos por inducci\'on. Los casos bases son satisfechos. Observe que si $3^k$ tiene como d\'igito de las decenas a $a$ y como d\'igito de las unidades a $b$ entonces $a$ es par por hip\'otesis de inducci\'on. $3^{k+1} $ tiene como d\'igito de las decenas al \'digito de las unidades de $3*a$ mas el d\'igito de las decenas de $3*b$. Note que estos dos d\'igitos mencionados anteriormente son siempre pares, puesto que $b\in \{1, 3, 7, 9\} \implies 3b \in \{3, 9, 21, 27\}$ y $a$ es par. Eso concluye la inducci\'on
\end{sol}


\subsection{Clases de residuos}

\begin{sol}
	Observe los siguientes conjuntos: $\{0\}, \{1, -1\}, \{2, -2\}, \{3, -3\}, \{4, -4\}, \{5, -5\}$. Son seis conjuntos, y representan todos los residuos m\'odulo $11$, agrupados. Luego, como hay 7 numeros, algunos de los n\'umeros ( o sus reciprocos aditivos [el reciproco aditivo de a es -a])  caeran en el mismo conjunto, por palomar. Esto implica que hay dos cuya suma o diferencia es m\'ultiplo de 11
\end{sol}


\begin{sol}
	Aplique m\'odulo 4 a la ecuaci\'on, lo cual implica que $1 + 2 \equiv 1 \mod 4$. Contradicci\'on.
\end{sol}

\begin{sol}
	Note que $n^3-1 = (n-1)(n^2+n+1)$. Por el algoritmo de euclides, $(n+1, n^2+n+1) = 1$. Luego, si $n+1 | n^3-1 \implies n+1|n-1$. Lo cual es una contradicci\'on, a menos que $n = 1$.
\end{sol}

\begin{sol}
	Veamos que $n = 1, 2$ satisfacen, luego podemos suponer que  $n> 2$. Observe que $2n-1| n^3+1 \iff (2n-1, n^3-1) = 2n-1, (\bullet, \bullet)$ denotando el m\'aximo com\'un divisor. Adem\'as, observe que $(a,bc) = (a,b)\cdot (a,c)$, para cualesquiera enteros positivos $a, b, c$. Vea que $(2n-1, n^3+1) = (2n-1, n+1)\cdot (2n-1, n^2+n-1) \leq 3\cdot (2n-1, n^2-3n+2) = 3\cdot (2n-1, n-1)\cdot(2n-1, n-2) = 3$. Donde usamos el algoritmo de euclides $(a,b) = (a,a-b)$. Luego $n>2 \implies 2n-1 > 3$ y por lo tanto $ (2n-1, n^3+1) \neq 2n-1$. 
\end{sol}


\subsection{Divisi\'on Modular}
\begin{sol}
	Procedemos por absurdo. Sea $a$ un divisor de cero invertible en $\mathbb{Z}_{n}$. Luego existen $b$ y $c$ tal que $(a\cdot b \equiv 1) \land (a\cdot c \equiv 0)$. Tenemos que $n|a\cdot c \implies n = (n, a\cdot c) = (n,a)\cdot (n,c) = (n,c)$. Note que $(n,a) = 1 $por el teorema 1.1 de la secci\'on 1.3. Es decir $n|c$. Pero eso es una contradicci\'on, por la definici\'on de divisor del cero.
\end{sol}

\begin{sol}
	$(\impliedby)$ Es claro debido a la proposici\'on 1.5 de la secci\'on 1.1.
	$(\implies)$ Multiplique ambos lados de la ecuaci\'on por $\frac{1}{a}$. (existe debido al teorema 1.1).
\end{sol}

\begin{sol}
	Sea $q = (n,a)>1$.Entonces $k= \frac{n}{q}$ es un entero con $k\not\equiv 0 \mod n$ debido a que $0<k<n$. Pero $a\cdot k \equiv 0 \mod n$. Claramente, $a$ es divisor del cero. 
\end{sol}

\begin{sol}
	Escoja el $k$ del ejercicio anterior, y $l=0$.
\end{sol}

\begin{sol}
	Defina $c = \frac{1}{a}\cdot \frac{1}{b}$ el candidato a inverso de $a\times b$. En efecto, $a\times b \times c \equiv a \cdot \frac{1}{a} \cdot b \cdot \frac{1}{b} \equiv 1 \mod n \implies a\times b \in (\mathbb{Z}_{n})^{*}$
\end{sol}

\begin{sol}
	\begin{align}
	7x &\equiv 3 \mod 15 \\
	13\cdot 7x &\equiv 3\cdot 13 \mod 15 \\
	x &\equiv 9 \mod 15
	\end{align}
	Para la segunda parte:
	\begin{align}
	3x &\equiv 7 \mod 15 \\
	5\cdot 3x &\equiv 7\cdot 3 \mod 15 \\
	0 &\equiv 6 \mod 15
	\end{align}
\end{sol}

\begin{sol}
	Si $a$ es su propio inverso, tenemos que $a^2 \equiv 1 \mod p$. Luego $(a-1)(a+1) \equiv 0 \mod p$. Como todos los residuos diferentes de cero son coprimos con $p$, sigue que $a \equiv \pm 1 \mod p$.
\end{sol}

