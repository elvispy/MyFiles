\section{EL teorema de Euler}

\begin{sol}
	Suponga que el conjunto no es un sistema completo de residuos m\'odulo $n$. Entonces existen $i, j$ tal que $i\neq j$, $a\cdot r_{i}+b \equiv a \cdot r_{j}+b \mod n \implies a \cdot r_{i} \equiv a \cdot r_{j} \implies r_{i} \equiv r_{j}$, que es una contradicci\'on
\end{sol}

\begin{sol}
	Copie la prueba del ejercicio anterior y use el ejercicio 1.19 de la secci\'on 1.3
\end{sol}

\begin{sol}
	$a\cdot b^{p} - b \cdot a^{p} = ab(b^{p-1}-a^{p-1}) \equiv 0 $ debido a que si uno de los dos numeros $a$ o $b$ son multiplos de $p$, el resultado final lo se\'a. Caso contraraio, el teorema de fermat garante que $a^{p-1} \equiv b^{p-1}$. 
\end{sol}

\begin{sol}
	$p^{8} \equiv 1 \mod 240 \iff (p-1)(p+1)(p^2+1)(p^4+1) \equiv 0 \mod 240$. Veamos que todos los factores de la parte izquierda de la euivalencia son pares, luego $2^4 | p^8-1$. Ademas, $(p+1)$ o $(p-1)$ son m\'ultiplos de 3, luego $3|p^8-1$. Falta probar que $5|p^8-1$. Suponamos que $p \not \equiv \pm 1 \mod 5$. Entonces $p \equiv 2$ o $3 \mod 5$. En cualquier caso, $p^2+1 \equiv 0 \mod 5$. Por lo que el resultado sigue.
\end{sol}

\begin{sol}
	Tome $m = \varphi(b)-1, n = \varphi(a)-1$. Entonces $a^m +b^n \equiv b^n \equiv 1 \mod a$. Del mismo modo, $a^m + b^n \equiv 1 \mod b$. Es decir, $(a| a^m+b^n-1) \land (b| a^m+b^n-1) \implies ab | a^m+b^n-1 \implies a^m+b^n \equiv 1 \mod ab$.
\end{sol}

\begin{sol}
	Supongamos que $p$ no es primo. Es decir, $p = a\cdot b$ con $1<a, b<p$. Luego, $ab | (p-1)! \implies p | (p-1)! \implies (p-1)! \equiv 0 \mod p$. Contradicci\'on.
\end{sol}

\begin{sol}
	Supongamos que si. Primero observemos que podemos suponer que $a_{p} \equiv b_{p} \equiv 0 \mod p$ (Ya que solo debe haber un cero m\'odulo p). Entonces $1 =(-1)^2=(p-1)!\cdot (p-1)! =(\prod_{i=1}^{p-1}a_{i})\cdot (\prod_{i=1}^{p-1}b_{i}) = \prod_{i =1}^{p-1} a_{i}b_{i} \equiv (p-1)! \equiv -1 \mod p$. Contradicci\'on
\end{sol}

\begin{sol}
	Veamos que con $4$ $2's$ y $1484$ $1's$ se puede lograr el cometido en $(a)$. Para $(b)$, note que $x^{7} \equiv x \mod 7$. Luego, $1998 \equiv \sum_{x} x^{7} \equiv \sum_{x} x \equiv 1492 \mod 7$, pero $1998 \not\equiv 1492 \mod 7$.
\end{sol}

\begin{sol}
	Solucion en proceso.
\end{sol}

\begin{sol}
	\begin{enumerate}[a.]
		\item Observe que $2 \equiv 2 \mod 7, 2^2 \equiv 4 \mod 7, 2^3 \equiv 1 \mod 7$, y el ciclo se repite. Luego, solo para $n \equiv 1 \mod 3 $ se tiene que $2^n \equiv 1 \mod 7$.
		\item$ 2^1 +1 \equiv 3 \mod 7, 2^2+1 \equiv 5 \mod 7, 2^3 +1 \equiv 2 \mod 7, 2^4+1 \equiv 3 \mod 7. $ El ciclo se repite debido a que $2^k \equiv 2^{k+3} \mod 7$. Como en el primer ciclo no hubo ning\'un cero, tenemos que la ecuaci\'on $2^n +1 \equiv 0 \mod 7 $ no tiene soluciones.
	\end{enumerate}
\end{sol}

\begin{sol}
	Observe que $\forall k \geq 0, m $ y $ m\cdot 10^{k}$ tienen la misma suma de sus d\'igitos. Por lo tanto, podemos asumir que $(s, 10) = 1$. Por el teorema de Euler, obtenemos que $10^{\varphi(s)\cdot k } \equiv 1 \mod s, \forall k \geq 1$. Si definimos $n = \sum_{i=1}^{s} 10^{\varphi(s) \cdot i}$, vemos que $n$ satisface las propiedades del problema.
\end{sol}

\begin{sol}
	Repita la prueba del ejercicio 4.2. Entonces, vea que el sistema es equivalente a $(p-1)! \equiv (p-1) \mod p, (p-1)! \equiv p-1 \mod (p-1)$. Lo cual es trivialmente verdadero.
\end{sol}

\begin{sol}
<<<<<<< HEAD
	Solucion en proceso
\end{sol}

\begin{sol}
	Solucion en proceso
\end{sol}

\begin{sol}
	Probaremos que siempre existe un $k$ para el cual $a_{k}$ no es coprimo con $p$, sea cual sea $p$ primo. Para eso, veamos que $2|a_{1}, 3|a_{2}$, por lo que podemos asumir que $(p, 6) = 1$. Por lo que 
	\begin{align}
	a_{p-2} &\equiv 2^{p-2} + 3^{p-2} + 6^{p-2} -1 \equiv 0 \iff \\
	\frac{1}{2}+\frac{1}{3}+\frac{1}{6}-1 &\equiv 0 \iff \\
	6 \cdot ( \frac{1}{2}+\frac{1}{3}+\frac{1}{6}-1) &\equiv 6 \cdot 0 \iff \\
	3 + 2 + 1 - 6 &\equiv 0
	\end{align}
	Lo cual es verdad. En la segunda l\'inea se us\'o el pequeño teorema de Fermat: $a^{\varphi(p)-1} \equiv \frac{1}{a} \mod p$.
\end{sol}

\begin{sol}
	Procederemos por inducci\'on. Suponga que hemos escogido los primeros $n$ elementos ($e_{i} = 2 ^{v_{i}} - 3$) de la lista, y sea $M_{n} = mcm(e_{1}, e_{2} \cdots e_{n})$. Entonces, utilizando el criterio de euler, vemos que $(2_{v_{i}}-3, 2^{k}-3) = (2^{v_{i}}-3, 2^{v_{i}}(2^{k-v_{i}}-1)) = (2^{v_{i}}-3, 2^{k-v_{i}}-1)$. Observe el siguiente sistema
\end{sol}


