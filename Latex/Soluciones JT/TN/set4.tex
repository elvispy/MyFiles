\section{Congruencias Lineales}

\begin{sol}
	\begin{align}
	4x +20 &\equiv 27x -1 \mod 15 \\
	21 & \equiv 23x \mod 15 \\
	21 \cdot 2 &\equiv 46x \mod 15 \\
	2 &\equiv x \mod 15
	\end{align}
\end{sol}

\begin{sol}
	$ a \equiv b \mod n \iff  n|a-b \iff p_{i}^{e_{i}}|a-b, \forall i \iff a\equiv b \mod p_{i}^{e_{i}}$
 
\end{sol}

\begin{sol}
	Analicemos la tercera ecuaci\'on. $4x equiv 20 \mod 12$ es equivalente a $(4x \equiv 20 \mod 4) \land (4x \equiv 20 \mod 3)$ que a su vez es equivalente a $x \equiv 2 \mod 3$. Entonces la cuarta ecuaci\'on puede ser ignorada, ya que es equivalente a la tercera ecuaci\'on. La segunda ecuaci\'on es $2x \equiv 8 \mod 4 \implies x \equiv 0 mod 2$. La primera ecuaci\'on dice $x \equiv 0 \mod 5$. Juntando todo da $x \equiv 20 \mod 30$. 
\end{sol}

\begin{sol}
	Tome $s = 1848$.
	Vea que $2011 $ es primo. Luego $2^{2000} \equiv \frac{1}{2^{10}} \cdot 2^{2010} \equiv \frac{1}{2^{10}} \cdot 2^{\varphi(2011)} \equiv \frac{1}{2^{10}} \equiv \frac{1}{1024} \equiv 1848 \mod 2011$.
\end{sol}


\begin{sol}
	Un n\'umero es autoreplicante si $n^2 \equiv n \mod 10000\implies n(n-1) \equiv 0 \mod 10000$. Como $n $ y $n-1$ son coprimos, tenemos que $n = 0, 1 \mod 10000$. Es decir, $n = 10000\cdot k + p, p \in \{0, 1\}$. Observe que ning\'un n\'umero d esta forma est\'a entre $1000$ y $9999$. 
\end{sol}

\begin{sol}
	Sea $\{p_{i}\}_{i\in \mathbb{N}}$ una enumeraci\'on de los primos. Considere el sistema de ecuaciones
	\begin{align}
	x &\equiv -1r \mod p_{1}^k \\
	x&\equiv -2r \mod p_{2} ^k \\
	&\vdots \hspace{1cm} \vdots \\
	x &\equiv -rn \mod p_{n}^{k}   
	\end{align}
	Debido a que todos los m\'odulos son coprimos, podemos aplicar el TCR. Entonces existe x una soluci\'on al sistema de congruencias. Luego, tenemos una progresi\'on aritm\'etica $x+r, x+2r, \dots x+rn$ tal que $p_{i}^{k} | x+ir, \forall i$.
\end{sol}

\begin{sol}
	Solucion pendiente
\end{sol}

\begin{sol}
	Sea $\{p_{i}\}_{i\in \mathbb{N}}$ una enumeraci\'on de los primos. Sea $q_{i} = \prod_{j=1}^{n} p_{(i-1)\cdot n + j }$. Considere el sistema de ecuaciones
	\begin{align}
	x &\equiv -1 \mod q_{1} \\
	x&\equiv -2 \mod q_{2} \\
	&\vdots \hspace{1cm} \vdots \\
	x &\equiv -k \mod q_{n}   
	\end{align}
	ebido a que todos los m\'odulos son coprimos, podemos aplicar el TCR. Entonces existe x una soluci\'on al sistema de congruencias. Luego, tenemos una progresi\'on aritm\'etica de raz\'on 1 tal que $q_{i} |x+i$. Es decir, $x+i$ tiene al menos $n$ primos distintos en su desomposici\'on 
\end{sol}

\begin{sol}
	Solucion en proceso
\end{sol}

\begin{sol}
	Observe primero que los residuos cuadr\'aticos m\'odulo $8$ son $0, 1, 4$. Como los cuadrados deben ser de n\'umeros impares, vemos que la suma de cinco (o nueve) elementos consecutios es congruente a $1$ m\'odulo $8$. Sea $\{a_{i}\}_{i \leq 100}$ la secuencia. Entonces tome $9$ elementos consecutivos, y tome un subconjunto de $5$ elementos consecutivos de \'este. Si substraemos el primer conjunto del segundo, tenemos que $a+b+c+d = p^2-q^2 = (p-q)(p+q) \equiv 0 \mod 8$ debido a que tanto $p$ como $q$ son impares. Podemos tomar estos cuatro elementos $a, b, c, d$ como siendo consecutivos
	\begin{equation}
	v_{1}, \hspace{0.2cm} v_{2}, \hspace{0.2cm} v_{3}, \hspace{0.2cm} v_{4}, \hspace{0.2cm} v_{5}, \hspace{0.2cm} a, \hspace{0.2cm} b, \hspace{0.2cm} c, \hspace{0.2cm} d, \hspace{0.2cm} e
	\end{equation}
	
	Por ejemplo, arriba $v_{1}+v_{2}+v_{3}+v_{4}+v_{5} $ es un cuadrado perfecto y tambien $v_{1}+v_{2}+v_{3}+v_{4}+v_{5} +a+b+c+d$. Sigue que $a+b+c+d \equiv 0 \mod 8$. Sigue que $v_{1} \equiv e \equiv 1 \mod 8$. Podemos transladar este raciocinio hacia la derecha y concluir que $a \equiv 1 \mod 8$. Asi, $a_{i} \equiv 1 \mod 8, \forall 10\leq i \leq 90$. Pero entonces tenemos que $a+b+c+d \equiv 4 \mod 8$, lo cual es una contradicci\'on a una afirmaci\'on hecha mas arriba.
\end{sol}

\begin{sol}
	Sea $\{p_{i}\}_{i\in \mathbb{N}}$ una enumeraci\'on de los primos. Escriba $a_{i} = \prod_{j\geq1} p_{i}^{e_{ij}}$.% Defina $c_{j} = mcm(e_{1j}, e_{2j}, \dots e_{nj})$. Note que $c_{j} = 1$ para $j$ suficientemente grande, debido a que $a_{i} < \infty \forall i$.
	 Escoja $n$ numeros primos suficientemente grandes ( $q_{i}, i\leq n$). Considere los siguientes sistemas de congruencias, para cada $j$:
	\begin{align}
	v_{j} &\equiv -e_{1j} \mod q_{1} \\
	v_{j} & \equiv -e_{2j} \mod q_{2} \\
	\vdots & \hspace{1cm} \vdots \\
	v_{j} & \equiv -e_{nj} \mod q_{n}
	\end{align}
	Como los m\'odulos son todos coprimos, podemos aplicar el TCR a cada sistema. Finalmente, defiamos $b = \prod_{j\geq 1} p_{j}^{v_{j}}$. Tenemos que para $j$ suficientemente grande, $v_{j} = 0$, debido a que $a_{i} < \infty, \forall i$. Tenemos enntonces el conjunto $\{ba_{1}, ba_{2}, ba_{3} \dots ba_{n}\} = \{  \prod_{j\geq1} p_{j}^{e_{1j}+v_{j}}, \prod_{j\geq1} p_{j}^{e_{2j} + v_{j}}, \dots \prod_{j\geq1} p_{j}^{e_{nj}+v_{j}}  \}$. Note que $v_{j} +e_{ij} \equiv 0 \mod q_{i}$, entonces $\{ba_{1}, ba_{2}, ba_{3} \dots ba_{n}\} = \{ \prod_{j\geq1} p_{j}^{k_{1j}\cdot q_{1}}, \prod_{j\geq1} p_{j}^{k_{2j} \cdot q_{2}} , \cdots, \prod_{j\geq1} p_{j}^{k_{nj}\cdot q_{n}}  \} = \{ s_{1}^{q_{1}}, s_{2}^{q_{2}}, \cdots s_{n}^{q_{n} } \}$.
\end{sol}