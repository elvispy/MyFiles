\section{Desigualdad de Cauchy-Schwarz}

\begin{sol}
	\begin{align}
	2\bigg(\frac{1}{a+b}+\frac{1}{a+c}+\frac{1}{c+b}\bigg) &\geq \frac{9}{a+b+c} \iff \\
	\frac{2(a+b+c)}{3} &\geq \frac{3}{\frac{1}{a+b}+\frac{1}{a+c}+\frac{1}{c+b}} \iff \\
	\frac{(a+b)+(b+c)+(c+a)}{3} &\geq \frac{3}{\frac{1}{a+b}+\frac{1}{a+c}+\frac{1}{c+b}}
	\end{align}
	Que sigue por MA-MG
\end{sol}

\begin{sol}
	Veamos que la desigualdad del problema es equivalente a
	\begin{center}
		$$ (\sum_{i=1}^{n} b_{i})(\sum_{i=1}^{n} \frac{a_{i}^{2}}{b_{i}})\geq (\sum_{i=1}^{n} a_{i})^{2} $$
	\end{center}
	
	Lo cual sigue de inmediato al usar CS con $(x_{i})_{i=1}^{n} = (\sqrt{b_{i}})_{i=1}^{n}, (y_{i})_{i=1}^{n} = (\frac{a_{i}}{\sqrt{b_{i}}})_{i=1}^{n}$. Como se ha usado \'unicamente la desigualdad CS, se aplica su criterio de igualdad, es decir, $a_{i} = \lambda b_{i}$.
\end{sol}

\begin{sol}
	Este ejercicio es un caso espec\'ifico del ejercicio siguiente con $n =4$.
\end{sol}

\begin{sol}
	\textbf{Notaci\'on}: si pongo una prima en la variable me refiero a la variable del ejercicio 6.2
	Usando el ejercicio 6.2, con $b'_{i} = a_{i} +b_{i}, a'_{i} = a_{i}$, tenemos que
	\begin{align}
		 \left(\sum_{i=1}^{n} b'_{i}\right)\left(\sum_{i=1}^{n} \frac{\left(a'_{i}\right)^{2}}{b'_{i}}\right)&\geq \left(\sum_{i=1}^{n} a'_{i}\right)^{2}  \implies \\		 
		 \left(2\sum_{i=1}^{n} a_{i}\right)\left(\sum_{i=1}^{n} \frac{a_{i}^{2}}{a_{i} + b_{i}}\right)&\geq \left(\sum_{i=1}^{n} a_{i}\right)^{2} \implies \\
		 \sum_{i=1}^{n} \frac{a_{i}^{2}}{a_{i} + b_{i}} &\geq \frac{1}{2}\sum_{i=1}^{n} a_{i}
	\end{align}
\end{sol}

\begin{sol}
	El resultado sigue aplicando CS con $(x_{i})_{i=1}^{n} = (x_{i})_{i=1}^{n}, (y_{i})_{i=1}^{n} = (\frac{1}{n})_{i=1}^{n}$
\end{sol}

\begin{sol}
	Elevando al cuadrado ambos lados queda: $4(a+b+c) + 3 + 2(\sqrt{4a+1}\sqrt{4b+1}+\sqrt{4b+1}\sqrt{4c+1}+\sqrt{4a+1}\sqrt{4c+1}) \leq 21 \implies \sqrt{4a+1}\sqrt{4b+1}+\sqrt{4b+1}\sqrt{4c+1}+\sqrt{4a+1}\sqrt{4c+1} \leq 7$. Ahora aplicamos CS con $(x_{1}, x_{2}, x_{3}) = (y_{1}, y_{2}, y_{3}) = (\sqrt{4a+1}, \sqrt{4b+1}, \sqrt{4c+1})$, el resultado sigue.
\end{sol}

\begin{sol}
	Solucion pendiente.
\end{sol}

\begin{sol}
	Solucion pendiente.
\end{sol}