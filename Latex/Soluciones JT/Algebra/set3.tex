\section{Polinomios en una variable}

\begin{sol}
	Observe que si $a \in \mathbb{R}$, $z, y \in \mathbb{C}$, entonces $\hat{axy} = a\cdot\overline{z}\cdot \overline{y}$. Sigue que $F(z) = \overline{F(\overline{z})}$. Por lo tanto, $F(z) = 0 \iff F(\overline{z}) = 0$. 
\end{sol}

\begin{sol}
	Observe simplemente que $F(1) < 0, F(3) > 0, F(5) <0$. Como $ deg(F) = 4$, obtenemos que por el teorema de valor intermediario ($F$ es un polinomio, y por lo tanto es una funci\'on continua), obtenemos que hay un cero en cada uno de los intervalos $(-\infty, 1), (1, 3), (3, 5), (5, \infty)$.
\end{sol}

\begin{sol}
	Definamos $Q(x) = (x+1)F(x)-x$. Sigue que $deg(Q) = n+1$, y que $Q(x) \equiv 0 \forall x = 0, 1, 2 \dots n$. Por el teorema fundamental del \'algebra, tenemos que $Q(x) = x(x-1)(x-2)(x-3)\cdots (x-n)$. Por lo tanto $Q(n+1) = (n+1)! = (n+2)F(n+1)-n-1 \implies F(n+1) = \frac{(n+1)!-(n+1)}{n+2}$.
\end{sol}

\begin{sol}
	Suponiendo que $a \geq 0$, podemos usar las relaciones de cardano vietta para concluir que $r_{1} \cdot r_{2} = \frac{c}{a}, r_{1}+r_{2} = -\frac{b}{a}$. Ahora simplemente observe que $r_{1}^{2}+r_{2}^{2} = (r_{1}+r_{2})^{2}-2r_{1} \cdot r_{2}$, y $r_{1}^{3}+r_{2}^{3} = (r_{1}+r_{2})^{3}-3\cdot r_{1}r_{2} \cdot (r_{1}+r_{2})$.
\end{sol}

\begin{sol}
	Sean $x, y$ las raices del polinomio. Por cardano Vietta obtenemos que $x\cdot y = b+1, x+y = -a$. 
\end{sol}

\begin{sol}
	Definamos $F(x) = a_{3}x^{3}+a_{2}x^{2}+a_{1}x+a_{0}$. Por cardano-Vietta, obtenemos que $r_{1}r_{2}r_{3} = a_{0}$, $a_{2} = r_{1}+r_{2}+r_{3}$. Es decir, $\frac{a_{2}}{a_{0}} = \frac{1}{r_{1}r_{2}}+\frac{1}{r_{2}r_{3}}+\frac{1}{r_{1}r_{3}}$. La condici\'on del problema puede ser reescrita como $\frac{a_{2}}{2}+2a_{0} = 1000a_{0} \implies a_{2} = 1996 a_{0} \implies \frac{a_{2}}{a_{0}} = 1996$.
\end{sol}