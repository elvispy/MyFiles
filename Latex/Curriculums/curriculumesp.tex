%%%%%%%%%%%%%%%%%%%%%%%%%%%%%%%%%%%%%%%%%%%%%%%%%%%%%%%%%%%%%%%%%%%%%%%%
%%%%%%%%%%%%%%%%%%%%%% Simple LaTeX CV Template %%%%%%%%%%%%%%%%%%%%%%%%
%%%%%%%%%%%%%%%%%%%%%%%%%%%%%%%%%%%%%%%%%%%%%%%%%%%%%%%%%%%%%%%%%%%%%%%%

%%%%%%%%%%%%%%%%%%%%%%%%%%%%%%%%%%%%%%%%%%%%%%%%%%%%%%%%%%%%%%%%%%%%%%%%
%% NOTE: If you find that it says                                     %%
%%                                                                    %%
%%                           1 of ??                                  %%
%%                                                                    %%
%% at the bottom of your first page, this means that the AUX file     %%
%% was not available when you ran LaTeX on this source. Simply RERUN  %%
%% LaTeX to get the ``??'' replaced with the number of the last page  %%
%% of the document. The AUX file will be generated on the first run   %%
%% of LaTeX and used on the second run to fill in all of the          %%
%% references.                                                        %%
%%%%%%%%%%%%%%%%%%%%%%%%%%%%%%%%%%%%%%%%%%%%%%%%%%%%%%%%%%%%%%%%%%%%%%%%

%%%%%%%%%%%%%%%%%%%%%%%%%%%% Document Setup %%%%%%%%%%%%%%%%%%%%%%%%%%%%

% Don't like 10pt? Try 11pt or 12pt
\documentclass[10pt]{article}

% This is a helpful package that puts math inside length specifications
\usepackage{calc}

% Simpler bibsection for CV sections
% (thanks to natbib for inspiration)
\makeatletter
\newlength{\bibhang}
\setlength{\bibhang}{1em}
\newlength{\bibsep}
{\@listi \global\bibsep\itemsep \global\advance\bibsep by\parsep}
\newenvironment{bibsection}%
{\vspace{-\baselineskip}\begin{list}{}{%
			\setlength{\leftmargin}{\bibhang}%
			\setlength{\itemindent}{-\leftmargin}%
			\setlength{\itemsep}{\bibsep}%
			\setlength{\parsep}{\z@}%
			\setlength{\partopsep}{0pt}%
			\setlength{\topsep}{0pt}}}
	{\end{list}\vspace{-.6\baselineskip}}
\makeatother

% Layout: Puts the section titles on left side of page
\reversemarginpar

%
%         PAPER SIZE, PAGE NUMBER, AND DOCUMENT LAYOUT NOTES:
%
% The next \usepackage line changes the layout for CV style section
% headings as marginal notes. It also sets up the paper size as either
% letter or A4. By default, letter was used. If A4 paper is desired,
% comment out the letterpaper lines and uncomment the a4paper lines.
%
% As you can see, the margin widths and section title widths can be
% easily adjusted.
%
% ALSO: Notice that the includefoot option can be commented OUT in order
% to put the PAGE NUMBER *IN* the bottom margin. This will make the
% effective text area larger.
%
% IF YOU WISH TO REMOVE THE ``of LASTPAGE'' next to each page number,
% see the note about the +LP and -LP lines below. Comment out the +LP
% and uncomment the -LP.
%
% IF YOU WISH TO REMOVE PAGE NUMBERS, be sure that the includefoot line
% is uncommented and ALSO uncomment the \pagestyle{empty} a few lines
% below.
%

%% Use these lines for letter-sized paper
\usepackage[paper=letterpaper,
%includefoot, % Uncomment to put page number above margin
marginparwidth=1.2in,     % Length of section titles
marginparsep=.05in,       % Space between titles and text
margin=1in,               % 1 inch margins
includemp]{geometry}

%% Use these lines for A4-sized paper
%\usepackage[paper=a4paper,
%            %includefoot, % Uncomment to put page number above margin
%            marginparwidth=30.5mm,    % Length of section titles
%            marginparsep=1.5mm,       % Space between titles and text
%            margin=25mm,              % 25mm margins
%            includemp]{geometry}

%% More layout: Get rid of indenting throughout entire document
\setlength{\parindent}{0in}

%% This gives us fun enumeration environments. compactitem will be nice.
\usepackage{paralist}

%% Reference the last page in the page number
%
% NOTE: comment the +LP line and uncomment the -LP line to have page
%       numbers without the ``of ##'' last page reference)
%
% NOTE: uncomment the \pagestyle{empty} line to get rid of all page
%       numbers (make sure includefoot is commented out above)
%
\usepackage{fancyhdr,lastpage}
\pagestyle{fancy}
%\pagestyle{empty}      % Uncomment this to get rid of page numbers
\fancyhf{}\renewcommand{\headrulewidth}{0pt}
\fancyfootoffset{\marginparsep+\marginparwidth}
\newlength{\footpageshift}
\setlength{\footpageshift}
{0.5\textwidth+0.5\marginparsep+0.5\marginparwidth-2in}
\lfoot{\hspace{\footpageshift}%
	\parbox{4in}{\, \hfill %
		\arabic{page} of \protect\pageref*{LastPage} % +LP
		%                    \arabic{page}                               % -LP
		\hfill \,}}

% Finally, give us PDF bookmarks
\usepackage{color,hyperref}
\definecolor{darkblue}{rgb}{0.0,0.0,0.7}
\hypersetup{colorlinks,breaklinks,
	linkcolor=darkblue,urlcolor=darkblue,
	anchorcolor=darkblue,citecolor=darkblue}

\usepackage[latin1]{inputenc} %Codificacion iso-8889-1

%%%%%%%%%%%%%%%%%%%%%%%% End Document Setup %%%%%%%%%%%%%%%%%%%%%%%%%%%%


%%%%%%%%%%%%%%%%%%%%%%%%%%% Helper Commands %%%%%%%%%%%%%%%%%%%%%%%%%%%%

% The title (name) with a horizontal rule under it
% (optional argument typesets an object right-justified across from name
%  as well)
%
% Usage: \makeheading{name}
%        OR
%        \makeheading[right_object]{name}
%
% Place at top of document. It should be the first thing.
% If ``right_object'' is provided in the square-braced optional
% argument, it will be right justified on the same line as ``name'' at
% the top of the CV. For example:
%
%       \makeheading[\emph{Curriculum vitae}]{Your Name}
%
% will put an emphasized ``Curriculum vitae'' at the top of the document
% as a title. Likewise, a picture could be included:
%
%   \makeheading[\includegraphics[height=1.5in]{my_picutre}]{Your Name}
%
% the picture will be flush right across from the name.
\newcommand{\makeheading}[2][]%
{\hspace*{-\marginparsep minus \marginparwidth}%
	\begin{minipage}[t]{\textwidth+\marginparwidth+\marginparsep}%
		{\large \bfseries #2 \hfill #1}\\[-0.15\baselineskip]%
		\rule{\columnwidth}{1pt}%
	\end{minipage}}
	
	% The section headings
	%
	% Usage: \section{section name}
	%
	% Follow this section IMMEDIATELY with the first line of the section
	% text. Do not put whitespace in between. That is, do this:
	%
	%       \section{My Information}
	%       Here is my information.
	%
	% and NOT this:
	%
	%       \section{My Information}
	%
	%       Here is my information.
	%
	% Otherwise the top of the section header will not line up with the top
	% of the section. Of course, using a single comment character (%) on
	% empty lines allows for the function of the first example with the
	% readability of the second example.
	\renewcommand{\section}[2]%
	{\pagebreak[3]\vspace{1.3\baselineskip}%
		\phantomsection\addcontentsline{toc}{section}{#1}%
		\hspace{0in}%
		\marginpar{
			\raggedright \scshape #1}#2}
	
	% An itemize-style list with lots of space between items
	\newenvironment{outerlist}[1][\enskip\textbullet]%
	{\begin{itemize}[#1]}{\end{itemize}%
		\vspace{-.6\baselineskip}}
	
	% An environment IDENTICAL to outerlist that has better pre-list spacing
	% when used as the first thing in a \section
	\newenvironment{lonelist}[1][\enskip\textbullet]%
	{\vspace{-\baselineskip}\begin{list}{#1}{%
				\setlength{\partopsep}{0pt}%
				\setlength{\topsep}{0pt}}}
		{\end{list}\vspace{-.6\baselineskip}}
	
	% An itemize-style list with little space between items
	\newenvironment{innerlist}[1][\enskip\textbullet]%
	{\begin{compactitem}[#1]}{\end{compactitem}}
	
	% An environment IDENTICAL to innerlist that has better pre-list spacing
	% when used as the first thing in a \section
	\newenvironment{loneinnerlist}[1][\enskip\textbullet]%
	{\vspace{-\baselineskip}\begin{compactitem}[#1]}
		{\end{compactitem}\vspace{-.6\baselineskip}}
	
	% To add some paragraph space between lines.
	% This also tells LaTeX to preferably break a page on one of these gaps
	% if there is a needed pagebreak nearby.
	\newcommand{\blankline}{\quad\pagebreak[3]}
	\newcommand{\halfblankline}{\quad\vspace{-0.5\baselineskip}\pagebreak[3]}
	
	% Uses hyperref to link DOI
	\newcommand\doilink[1]{\href{http://dx.doi.org/#1}{#1}}
	\newcommand\doi[1]{doi:\doilink{#1}}
	
	% For \url{SOME_URL}, links SOME_URL to the url SOME_URL
	\providecommand*\url[1]{\href{#1}{#1}}
	% Same as above, but pretty-prints SOME_URL in teletype fixed-width font
	\renewcommand*\url[1]{\href{#1}{\texttt{#1}}}
	
	% For \email{ADDRESS}, links ADDRESS to the url mailto:ADDRESS
	\providecommand*\email[1]{\href{mailto:#1}{#1}}
	% Same as above, but pretty-prints ADDRESS in teletype fixed-width font
	%\renewcommand*\email[1]{\href{mailto:#1}{\texttt{#1}}}
	
	%\providecommand\BibTeX{{\rm B\kern-.05em{\sc i\kern-.025em b}\kern-.08em
	%    T\kern-.1667em\lower.7ex\hbox{E}\kern-.125emX}}
	%\providecommand\BibTeX{{\rm B\kern-.05em{\sc i\kern-.025em b}\kern-.08em
	%    \TeX}}
	\providecommand\BibTeX{{B\kern-.05em{\sc i\kern-.025em b}\kern-.08em
			\TeX}}
	\providecommand\Matlab{\textsc{Matlab}}
	
	%%%%%%%%%%%%%%%%%%%%%%%% End Helper Commands %%%%%%%%%%%%%%%%%%%%%%%%%%%
	
	%%%%%%%%%%%%%%%%%%%%%%%%% Begin CV Document %%%%%%%%%%%%%%%%%%%%%%%%%%%%
	
	\begin{document}
		\makeheading{Elvis Alexander Aguero Vera}
		
		\section{Contacto}
		%
		% NOTE: Mind where the & separators and \\ breaks are in the following
		%       table.
		%
		% ALSO: \rcollength is the width of the right column of the table
		%       (adjust it to your liking; default is 1.85in).
		%
		\newlength{\rcollength}\setlength{\rcollength}{1.85in}%
		%
		\begin{tabular}[t]{@{}p{\textwidth-\rcollength}p{\rcollength}}
			\href{https://impa.br/}{Instituto de Matem\'{a}tica Pura y Aplicada (IMPA)}& \textit{Tel\'{e}fono:} +595986444069 \\
			Laboratorio FLUIDS & \ \ \ \ \ \ \ \ \ \ \ \ \, +5521972180469 \\
			
			Calle $1$ro de Mayo, 1032& \textit{e-mails:} \\  
		    Ciudad del Este, Alto Parana &    \email{elvisavfc65@gmail.com}\\
			Paraguay  &  \email{elvisavf@impa.br}
		\end{tabular}
		
		%\section{Objective}
		%
		
		%Placement in an academic position (i.e., faculty, postdoctoral, or
		%research scientist) that allows for advanced research in distributed
		%complex adaptive systems (i.e., modeling, analysis, design, and
		%verification) with a particular focus on the control of engineered
		%agents (e.g., for communications, control, software, and electronics
		%applications) and the analysis of biological phenomena (e.g.,
		%self-organization, ecological rationality)
		
		\section{Intereses de investigaci\'{o}n}
		%
		Matem\'{a}tica Aplicada, en particular Equaciones Diferenciales Parciales y Teoria Espectral de Operadores. \\
		
		\section{Educaci\'{o}n}
		%
		\href{http://www.impa.br/}{\textbf{Instituto Nacional de Matem\'{a}tica Pura y Aplicada (IMPA)}},
		Rio de Janeiro, Brasil.
		\begin{outerlist}
			
			%\item[] Ph.D.,
			%        \href{http://www.ece.osu.edu/}
			%             {Electrical and Computer Engineering},
			%             August 2010
			%        \begin{innerlist}
			%        \item Thesis Topic: \emph{Design and Analysis of Optimal
			%            Task-Processing Agents}
			%        \item Thesis Proposal: \emph{Cooperative Task Processing}
			%        \item Candidacy: \emph{Research
			%            Problems in Distributed Control for Energy Systems}
			%        \item Adviser:
			%              \href{http://www.ece.osu.edu/~passino/}
			%                   {Professor Kevin M.~Passino}
			%        \item Area of Study: Control Engineering
			%        \end{innerlist}
			
			\item[] Maestr\'{i}a en Matem\'{a}tica Aplicada Computacional, Febrero de 2019
			\begin{innerlist}
				\item Programa de posgrado en Matem\'{a}tica Aplicada Computacional.
				\item Orientador Acad\'{e}mico: \href{https://impa.br/page-pessoas/andre-nachbin/}{Profesor Andr\'{e} Nachbin}.
				\item Todos los requerimientos internos del IMPA han sido satisfechos el $1$ de Marzo de $2019$. Sin embargo, el IMPA tambi\'{e}n pide una carrera de grado para otorgar el diploma de m\'{a}ster.\\            
			\end{innerlist}
		\end{outerlist}
		
		{\textbf{Centro Regional de Educaci\'{o}n de Ciudad del Este}}, Ciudad del Este, Paraguay.
		\begin{outerlist}
			\item[] Bachiller T\'{e}cnico en Inform\'{a}tica, Diciembre de 2016.
			
		\end{outerlist}
		
		
		% Add a little space to nudge next ``Conference Publications'' marginpar
		% down to make room for tall ``Submitted Journal Publications''
		% marginpar. If there are enough submitted journal publications, this
		% space will not be needed (and should be removed).
		
		\section{Idiomas} \begin{loneinnerlist}
			
			\item Espa\~nol (Lengua Materna).
			\item Ingl\'{e}s: Lee bien, Habla bien, Escribe bien, Entiende bien.
			\item Portugues: Lee bien, Habla fluido, Escribe Bien, Entiende fluido.
			\item Guaran\'i: Lee bien, Habla bien, Escribe fluido, Entiende fluido.           
			
		\end{loneinnerlist}
		
		\vspace{0.1in}
		
		%
		%\section{Papers in Preparation} \begin{bibsection}
		%    \item Pavlic, T.P., K.M.~Passino. Distributed optimization under
		%        constraints: Pareto-optimal intelligent lighting.
		
		%    \item Pavlic, T.P. The ideal free distribution as degenerate form of
		%        nutrient-constrained optimization.
		%\end{bibsection}
		
		
		
		\section{Experiencia ense\~{n}ando}
		\href{http://impa.br/}{\textbf{Instituto Nacional de Matem\'{a}tica Pura y Aplicada}}, Rio de Janeiro, RJ, Brasil
		\begin{outerlist}
			\item[] \textbf{Asistente de profesor}\\
			\hfill Enero a Febrero, 2019
			\begin{innerlist}
				\item {An\'{a}lisis Funcional}\\
			\end{innerlist}
		\end{outerlist}
		%----
		%\halfblankline
		\href{http://www.omapa.org}{\textbf{OMAPA}},
		Ciudad del Este, Paraguay
		\begin{outerlist}
			\item[] \textbf{Entrenador de j\'{o}venes para Olimpiadas de Matem\'{a}ticas}\\
			2016-2018 (Diciembre de cada a\~no desde 2016)
			\begin{innerlist}
				\item {Curso de verano de matem\'{a}ticas para alumnos de secundaria.}
			\end{innerlist}
		\end{outerlist}
		%----
		%\section{Professional Experience}
		%
		
		%\section{Application Areas}
		%
		%Autonomous and Unmanned Vehicles, Flexible Manufacturing Systems,
		%Distributed Power Generation, Intelligent Lighting, Power Demand
		%Response, Microgrids, Smart Grids
		
		\section{Habilidades en Software} Lenguajes de programaci\'{o}n:
		%
		\begin{outerlist}
			\item R, Python, \Matlab.
		\end{outerlist}
		
		\halfblankline
		
		Software de productividad:
		%
		\begin{outerlist}
			\item \TeX{} (\LaTeX{})
			\item SQL
			\item Paquete tradicional de Ofim\'{a}tica (Word, Excel, Power Point).
		\end{outerlist}
		
		\halfblankline
		
		\section{Premios Internacionales}
		\href{http://www.imo-official.org/participant_r.aspx?id=25986}{\textbf{Olimpiada Internacional de Matem\'{a}tica (IMO)}}
		\begin{outerlist}
			\item 2016: Medalla de Bronce (Hong Kong, Hong Kong).
			\item 2015: Medalla de Bronce (Chiang Mai, Thailand).\\
		\end{outerlist}
		Olimpiada Iberoamericana de Matem\'{a}ticas
		\begin{outerlist}
			\item 2016: Medalla de Bronce (Antofagasta, Chile).
			\item 2015: Menci\'{o}n Honor\'{i}fica (Mayaguez, Puerto Rico).\\
		\end{outerlist}
		Olimpiada del Cono Sur
		\begin{outerlist}
			\item 2015: Medalla de Bronce (Temuco, Chile).
			\item 2014: Menci\'{o}n Honor\'{i}fica (Atl\'{a}ntida, Uruguay)\\
		\end{outerlist}
		Olimpiada de Mayo
		\begin{outerlist}
			\item 2014: Medalla de Oro
			\item 2012: Medalla de Plata
		\end{outerlist}
		
		\section{Premios Nacionales} Olimpiada de Matem\'{a}ticas Paraguayas
		\begin{outerlist}
			\item 2016: Medalla de Oro.
			\item 2015: Medalla de Oro.
			\item 2014: Medalla de Plata.
			\item 2013: Medalla de Oro.
			\item 2012: Medalla de Plata.
			\item 2011: Medalla de Plata.
			\item 2010: Medalla de Plata.\\
		\end{outerlist}
		Distinci\'{o}n local:
		\begin{outerlist}
			\item Medalla de orden al m\'{e}rito ``Domingo Martinez de Irala'', Asunci\'{o}n, 2017
			\item Ciudadano Ilustre de Ciuda del Este, Ciudad del Este, 2016.
		\end{outerlist}
		
		\section{Experiencia Laboral} 
		\href{http://www.pti.org.py}{\textbf{Parque Tecnol\'ogico Itaipu}},
		Ciudad del Este, Paraguay
		\begin{outerlist}
			\item[] \textit{Pasant\'{i}a remunerada}\\
			\textbf{ Julio 2019 - Actualmente}
			\begin{innerlist}
				\item {Pasante en el Centro de Innovaci\'on en Automatizaci\'on y Control (CIAC).} 
			\end{innerlist}
		\end{outerlist}
	
		\vspace{0.5cm}
		\href{http://www.omapa.org.py}{\textbf{OMAPA}},
		Ciudad del Este, Paraguay
		\begin{outerlist}
			\item[] \textit{Profesor particular}\\
			\textbf{ Abril 2019 - Actualmente}
			\begin{innerlist}
				\item {Profesor de jovenes de 15 a�os en el nivel de Pre-Avanzado para el Programa de J\'ovenes talentos de la Organizaci\'on Multidisciplinaria de Apoyo a Profesores y Alumnos.} 
			\end{innerlist}
		\end{outerlist}
		\vspace{0.5cm}
		\href{https://www.itaipu.gov.br/es}{\textbf{Itaipu Binacional}},
		Ciudad del Este, Paraguay
		\begin{outerlist}
			\item[] \textit{Pasant\'{i}a remunerada}\\
			\textbf{ Enero y Febrero de 2016}
			\begin{innerlist}
				\item {Puesto en la superintendencia de inform\'{a}tica (SI.GG Margen derecha), como ayudante en el mantenimiento de computadoras y asistencia t\'{e}cnica en Itaipu.} 
			\end{innerlist}
		\end{outerlist}	
		
	
		
		\section{Actividades Voluntarias}{Mentor\'{i}a a j\'{o}venes talentos del Alto Paran\'{a} en su entrenamiento para olimpiadas} \\ \\
		{ Participante en un proyecto de servicio social que consist\'{i}a en dotar de infraestructura colegios secundarios para crear una biblioteca digital interconectada} \\ \\
		{Voluntario en la organizaci\'{o}n no gubernamental AIGA (Asociaci\'{o}n Impulsora de Gobierno Abierto), que lucha por la transparencia y el uso correcto de Datos Abiertos, entre otros asuntos. }
		
		\section{Otras actividades}\href{http://www.omapa.org.py}{\textbf{OMAPA}},
		Asunci\'{o}n, Paraguay
		\begin{outerlist}
			\item[] \textit{Alumno del }\href{http://www.omapa.org/proyectos/iniciacion-cientifica/}
			{\textbf{Programa de Iniciaci\'{o}n cient\'{i}fica para J\'{o}venes talentos (JT)}}\\
			\textbf{2010-2016}
			\begin{innerlist}
				\item {Luego de recibir una medalla de plata en las Olimpiadas Matem\'{a}ticas en 2010, fui seleccionado para poder ser parte del programa, cumpliendo con el por 6 a\~{n}os acad\'{e}micos.}\\
			\end{innerlist}
			
		\end{outerlist}
		%
		
		\href{http://www.une.edu.py/web/index.php/conservatorio}{\textbf{Conservatorio Bellas Artes}},
		Ciudad del Este, Paraguay
		\begin{outerlist}
			\item[] \textit{Alumno}\\
			\textbf{2012-2016}
			\begin{innerlist}
				\item {Curso de Piano (anual y de verano), cinco a\~nos acad\'{e}micos.}
				\item {Curso anual de teor\'{i}a musical, cinco a\~nos acad\'{e}micos.}
				\item {Cursos de verano  de guitarra cl\'{a}sica (2012-2013)} \\
			\end{innerlist}
			
		\end{outerlist}
		%
		
		\section{Referencias Acad\'{e}micas}
		\href{http://nachbin.impa.br/Home}{\textbf{Prof. Andr\'e Nachbin}} (e-mail:~\href{mailto:nachbin@impa.br}{nachbin@impa.br})
		%
		\begin{innerlist}
			\item Profesor e investigador del \href{http://www.impa.br/}{IMPA}, Rio de Janeiro, Brasil.
			\item Andr\'{e} fue mi orientador acad\'{e}mico de la maestr\'{i}a.\\
		\end{innerlist}
		
		\halfblankline
		
		\href{http://w3.impa.br/~landim/}{\textbf{Prof. Claudio Landim}}
		(e-mail:~\href{mailto:landim@impa.br}{landim@impa.br})
		%
		\begin{innerlist}
			\item Profesor e investigador del \href{http://www.impa.br/}{IMPA}, Rio de Janeiro, Brasil.
			\item Claudio fue mi profesor de dos cursos de maestr\'{i}a, y adem\'{a}s yo fui su asistente en el curso de an\'{a}lisis funcional en el  \href{http://www.impa.br/}{IMPA}.\\
		\end{innerlist}
		
%		\halfblankline
%		
%		\href{http://w3.impa.br/~viana/}{\textbf{Prof. Marcelo Viana}}
%		(e-mail:~\href{mailto:viana@impa.br}{viana@impa.br})
%		%
%		\begin{innerlist}
%			\item Full Professor of Mathematics at \href{http://www.impa.br/}{IMPA}, Rio de Janeiro, Brazil.
%			\item Instructor of M.Sc. modules at \href{http://www.impa.br/}{IMPA}.\\
%		\end{innerlist}
		
		\halfblankline
		
		\href
		{https://www.researchgate.net/profile/Carlos_Galeano_Rios?ev=hdr_xprf&_sg=4JWHq4npTaFWm-nghSDooxePOQCGrJUuo7nF6rpzGFUMeIu-oTMc8eEw6f0TkhQjdhz4JwCieL8SQPj8tIgqopAf}
		{\textbf{Dr. Carlos Antonio Galeano Rios}}
		(e-mail:~\href{mailto:C.A.Galeano.Rios@bath.ac.uk}{C.A.Galeano.Rios@bath.ac.uk})
		%
		\begin{innerlist}
			%\item \href{https://researchportal.bath.ac.uk/en/persons/carlos-galeano-rios}{Investigador asociado} en el 
			\item Investigador asociado en el
			\href{https://www.bath.ac.uk/departments/department-of-mathematical-sciences/}{departamento de ciencias matem\'{a}ticas}. \href{https://www.bath.ac.uk/}{Universidad de bath}, Bath, Reino Unido.
			\item Mentor acad\'{e}mico.\\
		\end{innerlist}
		
		
	\end{document}
	%%%%%%%%%%%%%%%%%%%%%%%%%% End CV Document %%%%%%%%%%%%%%%%%%%%%%%%%%%%%
	
	%----------------------------------------------------------------------%
	% The following is copyright and licensing information for
	% redistribution of this LaTeX source code; it also includes a liability
	% statement. If this source code is not being redistributed to others,
	% it may be omitted. It has no effect on the function of the above code.
	%----------------------------------------------------------------------%
	% Copyright (c) 2007, 2008, 2009, 2010, 2011 by Theodore P. Pavlic
	%
	% Unless otherwise expressly stated, this work is licensed under the
	% Creative Commons Attribution-Noncommercial 3.0 United States License. To
	% view a copy of this license, visit
	% http://creativecommons.org/licenses/by-nc/3.0/us/ or send a letter to
	% Creative Commons, 171 Second Street, Suite 300, San Francisco,
	% California, 94105, USA.
	%
	% THE SOFTWARE IS PROVIDED "AS IS", WITHOUT WARRANTY OF ANY KIND, EXPRESS
	% OR IMPLIED, INCLUDING BUT NOT LIMITED TO THE WARRANTIES OF
	% MERCHANTABILITY, FITNESS FOR A PARTICULAR PURPOSE AND NONINFRINGEMENT.
	% IN NO EVENT SHALL THE AUTHORS OR COPYRIGHT HOLDERS BE LIABLE FOR ANY
	% CLAIM, DAMAGES OR OTHER LIABILITY, WHETHER IN AN ACTION OF CONTRACT,
	% TORT OR OTHERWISE, ARISING FROM, OUT OF OR IN CONNECTION WITH THE
	% SOFTWARE OR THE USE OR OTHER DEALINGS IN THE SOFTWARE.
	%----------------------------------------------------------------------%
